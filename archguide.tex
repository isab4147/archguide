\documentclass[dark, index]{Iart}

\usepackage{Ibash, kantlipsum}
\newcommand{\arch}{\color{theme}\fontspec{NotoSerif Nerd Font Mono} }

\title{Guia \textit{definitivo} de instalação do \textbf{\color{theme}Arch Linux} \arch}
\subtitle{Uma compilação de fontes e recomendações para o dia-a-dia do \textit{sys admin}.}
% TODO (maybe?): https://stackoverflow.com/questions/27555572/customizing-the-table-of-contents-with-two-colors-in-latex
\author{Isabella B. \email{isabellabdoamaral@usp.br} \& Juan M. \email{seu@email.com} \& walysu \email{seu@tbm.br}}



\begin{document}
	
	\maketitle
	
	\chapter*{Prefácio}

  Esse guia tem por objetivo ajudar aqueles que estão começando no Linux, especialmente no Arch Linux. A proposta é simples: reunir o conhecimento de usuários interessados e que querem contribuir para a comunidade e, por isso, esse guia é feito numa classe própria e simples, e está disponibilizado (com código fonte completo) no \href{github.com/isab4147/archguide}{github}.

  \section*{Por que Arch?}

  Caso você tenha caído aqui de paraquedas ou simplesmente ainda não esteja convencido de que deveria usar o Arch (muito difícil de instalar, instável, etc.), já adianto: \textbf{não vou ser eu a te convencer!}
  
  E também tenho um recado aos iniciantes motivados: sem a devida atenção o sistema pode quebrar muito rápido, e a instalação talvez não seja trivial (pro seu sistema, pelo menos\ldots).

  Mas, fora isso, existem inúmeras vantagens numa instalação Arch para um usuário intermediário-avançado, por exemplo:
  \begin{itemize}
    \item \textbf{Muita customização desde o primeiro contato}
      
      A instalação base é mínima (\hyperref[subs:kernelIntro]{nos limites do kernel}). 
    \item \textbf{O controle do usuário é amplo}

      Isso não necessariamente tem a ver com customização, mas o chamado \textit{Arch Way} é deixar o usuário nos limites de sua liberdade, adotando a filosofia do \textbf{RTFM}\footnote{Caso você não saiba, isso quer dizer \textit{Read The Fucking Manual}.}.
    \item \textbf{Muito educativo}
      
      Segue diretamente do último ponto abordado, e é a principal motivação desse guia: o Arch não é um Linux para usuários preguiçosos. Ele requer \textbf{tempo} e \textbf{interesse} ou, como diria uma amiga: é um pinguim carente. 
  \end{itemize}

  Se você é iniciante e acha que esse é o sistema para você: seja muito bem vindo e vamos \hyperref[part:install]{instalar o Arch!} Caso contrário, existe uma parte nesse guia dedicada à \hyperref[part:linuxSys]{discussão do sistema Linux} e \hyperref[chap:distros]{comparação com outras distros.}

  \vfill

  \begin{flushright}
    \itshape Isabella B.,\\
    27 de outubro, 2020
  \end{flushright}
	
	%\TODO{objetivos gerais etc...}
	
	\part{Instalação}
  \label{part:install}
	
	\chapter{Preparando o meio de instalação}
    Enfim, essa é a parte fácil, vamos nos preparar para a instalação. Existem alguns passos que, embora não sejam necessários seguem como recomendações para uma instalação mais \textit{suave}:

  \begin{itemize}
    \item Confira se o disco em que vai instalar possui tabela de partição GPT.
    \item 
  \end{itemize}
	
	\section{Baixe a iso}
	
	\section{Prepare o pen drive}
		
	\chapter{Bootando o PC}
	
  \chapter{Speedrun pros brabos}
  % TODO: conferir o prompt
  
  %firstnumber=auto, name=same
  %firstnumber=last
  %frame=trBL
  %frameround=fttt
\begin{lstlisting}
loadkeys br-abnt2
ls /sys/firmware/efi/efivars

# Para BIOS Legacy
fdisk -l
# se disklabel for GPT continue, se não
dd if=/dev/zero of=/dev/'disco' bs=4096 status=progress
gdisk /dev/'disco'
$ n
$ 
$ +1M
$ ef02
$ w

# sem cabo
# 1. celular
ip link show
ip link set 'seu-aparelho' up
dhcpcd -h 'seu-aparelho'
# 2. wifi
rfkill unblock all
[root@archiso root]# iwctl
[iwd]# station wlan0 connect 'rede'
password: 
[iwd]# quit

# teste de rede
ping -c 2 google.com
timedatectl set-ntp true

# particionamento
fdisk -l
fdisk /dev/'disco'
# d para deletar
# n para nova partição
# t para mudar o tipo
# p para listar as partições
# w para escrever as alterações
# q para cancelar

# dica:
# para instação Vanilla faça uma partição
# pro / e outra para home 
# para btrfs (avançado) faça uma partição
# única para ambos
# em ambos os casos faça uma partição para
#boot (tipo BIOS boot ─ código 4)
# e talvez swap também (tipo Linux Swap ─ código 19)

# formatação (para single boot)
mkfs.fat -F32 /dev/'pboot'
# para dual boot
mkfs.ext4 /dev/'pboot'
# caso tenha feito o swap
mkswap /dev/'pswap'
swapon /dev/'pswap'

# para btrfs
mkfs.btrfs -L 'label' /dev/'punique'

# para Vanilla
mkfs.ext4 /dev/'pslash'
mkfs.ext4 /dev/'phome'

# montagem Vanilla
mount /dev/'pslash'

# devboot = partição com bios boot
# devslash = partição do /
# devhome = partição do home
# devswap = partição do swap
mkfs.fat -F32 devboot
mkfs.ext4 devslash
mkfs.ext4 devhome
mkswap devswap
mount devslash /mnt
mkdir /mnt/{boot,home}
mount devhome /mnt/home

# para BIOS boot
mount devboot /mnt/boot

# para EFI boot
mkdir /mnt/boot/efi
mount devboot /mnt/boot/efi

# continuando
swapon devswap
# confira se montou tudo corretamente
lsblk

# para hardware mais novo
pacstrap /mnt base base-devel linux linux-firmware linux-headers

# para hardware mais antigo (ou para uma experiência mais estável)
pacstrap /mnt base base-devel linux-firmware linux-lts linux-lts-headers linux-lts-docs

genfstab -U -p /mnt >> /mnt/etc/fstab



# agora vamos entrar no sistema
arch-chroot /mnt
timedatectl set-timezone America/Sao_Paulo

ln -sf  /usr/share/zoneinfo/America/Sao_Paulo  /etc/localtime

hwclock --systohc
pacman -S nano
nano /etc/locale.gen

/etc/locale.gen
---------------
(descomente a linha do local q vc quer ─ ambos UTF8 e ISO)

# salve e saia
locale-gen
# lang -> lingua do local que você escolheu

echo LANG=lang.UTF-8 >> /etc/locale.conf
echo KEYMAP=br-abnt2 >> /etc/vconsole.conf
echo 'nome do pc' >> /etc/hostname
nano /etc/hosts

/etc/hosts
-------------
127.0.0.1 \t localhost \n
::1 \t localhost \n
127.0.1.1 \t 'nome do pc'.localdomain \t 'nome do pc'

# salve e saia
passwd
# digite a senha do root

pacman -S dosfstools mtools dialog sudo ntfs-3g
# se você conectou pelo wifi (ou celular) possui hardware mais novo
pacman -S networkmanager wireless_tools dhcpcd iwd
# -- hardware mais antigo
pacman -S networkmanager wireless_tools dhcpcd iwd wpa_supplicant netctl

# para processador intel
pacman -S intel-ucode
# para processador amd
pacman -S amd-ucode

# caso esteja instalando pela primeira vez
useradd -m -g users -G wheel 'nome do usuário'

# caso esteja reinstalando
useradd -g users -G wheel nome_do_usuario

usermod -d /home/nome_do_usuario -m nome_do_usuario

# após isso
passwd 'nome do usuário'
nano /etc/sudoers

/etc/sudoers
---------------
# Insult me
Defaults \t insults

%wheel ALL=(ALL) ALL

# salve e saia

#Para BIOS boot
pacman -S grub
grub-install --target=i386-pc --recheck /dev/sda

# OU
#Para EFI boot
pacman -S grub efibootmgr

grub-install --target=x86_64-efi --efi-directory=/boot/efi --bootloader-id=arch_grub --recheck

# após isso
grub-mkconfig -o /boot/grub/grub.cfg

# para dual boot
pacman -S os-prober
os-prober

# enfim
^D

shutdown now


# logue como seu user
sudo pacman -Sy xorg-server

# para placa de vídeo AMD
sudo pacman -S xf86-video-amdgpu
# ou pesquise para nvidia

# VANILLA
sudo pacman -S libgl mesa sddm plasma konsole dolphin alsa pulseaudio-alsa network-manager-applet

systemctl enable sddm

reboot

# logue no sistema
sudo pacman -S git 

git clone https://aur.archlinux.org/yay
cd yay
makepkg -si

sudo nano /etc/pacman.conf

/etc/pacman.conf
#descomente as duas linhas abaixo

// IMAGEM

# salve e saia

sudo pacman -Syyu

yay -S networkmanager-iwd

sudo systemctl enable NetworkManager
sudo systemctl enable iwd


Algumas packages recomendadas:
brave -> navegador; neofetch; cmatrix -> são legais
wps -> equivalente ao office, alternativamente use o libreoffice-fresh + libreoffice-fresh-pt-br
okular -> para pdfs
gthump -> para imagens
galculator -> calculadora legalzinha, alternativamente: kcalc
whatsapp, zoom, discord, stremio -> clientes das aplicações
gparted -> gerenciamento de disco
smplayer, vlc -> players
p7zip -> 7z
spectacle -> screenshots, alternativamente use maim + xclip + sxiv ou escrotum
kdeconnect -> conexão com celular (requer app)
emacs -> editor, alternativamente use o neovim
playonlinux -> para aplicações windows
man-db, man-pages, man-pages-pt_br -> páginas de manual para o arch
kdegraphics-thumbnailers -> para ver thumbs no dolphin

yay -S brave neofetch cmatrix wps-office wps-office-extension-portuguese-dictionary okular gthumb qbittorrent spectacle galculator whatsapp-nativefier smplayer discord gparted p7zip zoom vlc kdeconnect emacs man-db man-pages man-pages-pt_br kdegraphics-thumbnailers stremio

para spotify:
curl -sS https://download.spotify.com/debian/pubkey.gpg | gpg --import -
yay -S spotify

para latex:
yay -S texlive-bin texlive-core texlive-most texlive-lang tllocalmgr-git

http://www.hannahdee.eu/blog/?p=835
https://tex.stackexchange.com/questions/445521/latexindent-cant-locate-log-log4perl-pm-in-inc-you-may-need-to-install-the-l


para texstudio:
yay -S texstudio
sudo chmod 777 ~/.config/texstudio/texstudio.ini

fontes recomendadas:
yay -S adobe-source-han-sans-jp-fonts powerline-fonts ttf-baekmuk ttf-arphic-uming noto-fonts-emoji nerd-fonts-complete

fontes nice:
yay -S ttf-iosevka ttf-iosevka ttf-roboto ttf-raleway ttf-fira-mono

https://linuxwacom.github.io/


para o dolphin: 

balooctl disable
rm -rf ~/.local/share/baloo

https://github.com/vitalif/grive2/issues/287

mamutal91


https://linuxcommando.blogspot.com/2013/10/how-to-connect-to-wpawpa2-wifi-network.html?m=1

https://techbeat.in/2020/01/15/setup-arch-linux-on-dell-xps-13-7390.html

https://wiki.archlinux.org/index.php/installation_guide

https://diolinux.com.br/2019/07/como-instalar-arch-linux-tutorial-iniciantes.html

https://itsfoss.com/install-arch-linux/


VAR=$(cat ~/gdrive/texpackages.out)
yay -S ${VAR}

pac -Qqt

yay -Rcns (yay -Qqdt)

A find alternative: https://github.com/sharkdp/fd, alternativa pro find

https://github.com/ogham/exa, alternativa pro ls

https://github.com/sharkdp/bat, alternativa pro cat

https://github.com/BurntSushi/ripgrep, alternativa pro grep
https://fishshell.com/, alternativa pro Bash
https://github.com/phiresky/ripgrep-all

Mesma coisa q o ripgrep, só q pesquisa dentro de pdfs tbm

https://www.archlinux.org/news/arch-conf-2020-schedule/

https://github.com/elenapan/dotfiles

caso seu pen drive se torne inutilizável no Windows, faça:

win+r
diskpart
list disk
select disk num
clean
select disk num
create partition primary
select partition 1
active
format fs=fat32 quick

caso ainda não apareça, altere o caminho na ferramenta de gerenciamento de disco


https://wiki.archlinux.org/index.php/NVIDIA

https://wiki.archlinux.org/index.php/PRIME

https://rufus.ie/

https://www.archlinux.org/download/

https://www.majorgeeks.com/mg/getmirror/bootable_usb_test,1.html

https://wiki.archlinux.org/index.php/System_time#UTC_in_Windows

https://www.msi.com/Motherboard/support/H81M-E33#down-driver&Win10%2064

https://gist.github.com/ppartarr/175aa0c3416daf3baacde17f442f80e1

https://github.com/egara/arch-btrfs-installation

caso haja problemas nas chaves de uma package, importe desse servidor:

keyserver.ubuntu.com

e.g.:
tentamos instalar o powerpill
$ pacman -S powerpill
e no final temos o erro das chaves
...
[GPG]
key, could not be imported

portanto rodamos
gpg --keyserver keyserver.ubuntu.com --recv-keys key

isso é pra verificar suas opções de mount antes de fazer merda

sudo findmnt --verify --verbose


OPÇÕES DE MOUNT (para fstab)
troque relatime -> noatime nas partições de uso comum e no tmp
discard para ativar TRIM
autodefrag para fazer automaticamente no boot
compress=lzo como opção eficiente de defrag
https://btrfs.wiki.kernel.org/index.php/Manpage/btrfs(5)
https://mutschler.eu/linux/btrfs/
https://www.reddit.com/r/archlinux/comments/5zi34u/recommended_file_system_for_a_nvme_ssd_and_your/
https://www.reddit.com/r/openSUSE/comments/6u9gmo/performance_issues_related_with_btrfs/

noatime,ssd,discard,autodefrag,compress=lzo

dps no terminal:
btrfs quota disable partição

https://btrfs.wiki.kernel.org/index.php/SysadminGuide


Resolver problema do discord:

Baixa o discord pelo site
extrai o .tar.gz (de preferência com bsdtar -xf)
entre na pasta que tu extraiu
chmod +x postinst.sh
./postinst.sh
renomeia essa pasta para discord (de preferência usando o comando mv)
cd /lugar/do/arquivo
mv /nome/do/arquivo discord
sudo cp -r /lugar/do/arquivo/discord /opt
da um yay -S discord pra atualizar o lançador


SETUP DE UMA INSTALAÇÃO MAIS SEGURA:
# tente instalar o powerpill
# ele vai dar erro de chave, copie e cole a chave como abaixo:

gpg --keyserver keyserver.ubuntu.com --recv-keys EC3CBE7F607D11E663149E811D1F0DC78F173680

yay -S powerpill pacmatic

sudo nano /etc/powerpill/powerpill.json

/etc/powerpill/powerpill.json
------------
"aria2":{
	...
	# adicione a linha do print 
	# salva e sai
	
	yay -S fish z
	chsh -s /bin/fish
	
	# reloga
	curl -L https://get.oh-my.fish/ | fish
	omf install bobthefish
	git clone https://github.com/jbonjean/re-search.git
	omf install re-search
	cd re-search
	make
	mkdir ~/bin
	cp re-search ~/bin
	cp re_search.fish ~/.config/fish/functions
	echo bind \cr re_search >> ~/.config/fish/functions/fish_user_key_bindings.fish
	fish_config
	
	nano ~/.config/fish/config.fish 
	
	~/.config/fish/config.fish
	-----------
	
	alias pac=yay
	
	# pacmatic needs to be run as root: https://github.com/keenerd/pacmatic/issues/35
	alias pacmatic='sudo --preserve-env=pacman_program /usr/bin/pacmatic'
	
	## ATENÇÃO EM seu-user ABAIXO
	# Downgrade permissions as AUR helpers expect to be run as a non-root user. $UID is read-only in {ba,z}sh.
	function yay --wraps pacman_program --description 'alias yay=pacman_program'
	pacman_program="sudo -u seu-user$UID /usr/bin/yay --pacman powerpill" pacmatic $argv
	end
	
	set -gx PATH $PATH $HOME/bin
	
	set -g theme_date_timezone America/Sao_Paulo
	
	## salva e sai
	
	# agora é só reabrir o terminal
	
	pac -S rustup
	rustup install stable
	rustup default stable
	pac -S topgrade
	
	https://aur.archlinux.org/packages/?O=0&K=topgrade e https://github.com/r-darwish/topgrade
	
	o comando "topgrade -c" vai rodar todos os comandos responsaveis por atualização de pacotes/softwares, como, yay -Syu --devel e git pull --rebase, além de limpar o cache


https://wiki.archlinux.org/index.php/Reflector
https://wiki.archlinux.org/index.php/Xorg/Keyboard_configuration
https://wiki.archlinux.org/index.php/Touchpad_Synaptics


https://wiki.archlinux.org/index.php/GnuPG#Create_a_key_pair
https://wiki.archlinux.org/index.php/Pass



GnuPG
https://wiki.archlinux.org/index.php/Pacman/Tips_and_tricks


https://wiki.archlinux.org/index.php/Bluetooth_headset
https://unix.stackexchange.com/questions/255509/bluetooth-pairing-on-dual-boot-of-windows-linux-mint-ubuntu-stop-having-to-p


https://github.com/overlock1



/etc/X11/xorg.conf.d/30-touchpad.conf
--------------
Section "InputClass"
Identifier  "Touchpad"
Driver  "libinput"
MatchIsTouchpad "on"
Option "AccelProfile" "flat"
Option "AccelSpeed" "1"
Option "ClickMethod" "clickfinger"
Option "DisableWhileTyping" "true"
Option "HorizontalScrolling" "true"
Option "NaturalScrolling" "true"
Option "ScrollMethod" "twofinger"
Option "Tapping" "true"
Option "TappingDrag" "true"
Option "TappingDragLock" "true"
EndSection
https://wiki.archlinux.org/index.php/Libinput


https://jlk.fjfi.cvut.cz/arch/manpages/man/libinput.4


https://wiki.archlinux.org/index.php/Pacman/Package_signing#Cannot_import_keys


https://disconnected.systems/blog/archlinux-repo-in-aws-bucket/#wrapper-scripts


https://www.reddit.com/r/archlinux/comments/7v7g4w/managing_multiple_arch_linux_systems_with/

sudo su
fdisk -l
dd bs=4M if=/skjajsj/iso of=/dev/sdb status=progress oflag=sync

https://wiki.archlinux.org/index.php/System_time#UTC_in_Windows

ln -s /usr/share/X11/xorg.conf.d/40-libinput.conf /etc/X11/xorg.conf.d/40-libinput.conf

chsh -s /bin/fish

https://store.kde.org/p/1312658/

https://gist.github.com/ppartarr/175aa0c3416daf3baacde17f442f80e1


crie o
/etc/sysctl.d/99-sysctl.conf

kernel.sysrq=1

\end{lstlisting}

	\part{Entendendo o sistema Linux}
  \label{part:linuxSys}
  
  \chapter{Uma breve exploração das distribuições Linux}
  \label{chap:distros}

  \section{E o Arch nisso tudo?}
  
  \subsection{Customização do Kernel, compilando pacotes e a visão do Gentoo}
  \label{subs:kernelIntro}
\end{document}
